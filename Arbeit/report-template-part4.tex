\section{Bots und Mods} \label{question2} % Change title accordingly

Statistische Auswertungen des Datensatzes:\\
\textit{Ist es m{\"o}glich Bots und Moderatoren anhand des Datensatzes zu identifizieren?}\\
\textit{K{\"o}nnen wir dazu entsprechende Statistiken entwickeln und auswerten?}\\
\textbf{17.05.}\\
Vermutungen:\\
- viele Nutzer setzen nur wenige Pixel, gro{\"ss}e H{\"a}ufigkeit bei kleiner Pixelzahl\\
- Bots setzen konstant viele Pixel, leicht erh{\"o}hte H{{"a}ufigkeit bei gro{\"ss}er Pixelzahl\\
- Testdatensatz umfasst eine Stunde, maximal 12 Pixel pro Nutzer m{\"o}glich\\
Vorgehen:\\
- Nach Nutzern aggregierte Pixeldaten nochmal nach Pixelanzahl aggregieren\\
- Grafik \\
\textbf{30.05.}\\
Statistiken:\\
- meiste umk{\"a}mpfter Pixel: \textbf{Grafik} ev. in 3-Dimensionen\\
- meiste verwendete Farben: \textbf{Grafik}\\
\textbf{21.06.}\\
- Filterkriterien Bots: ca. 14.000 Bots im Testdatensatz\\
- Filter Moderatoren,  Welche Farben wurden verwendet?\\
\textbf{28.06.}\\
- Grafik Bots 2022\\
\textbf{05.07.}\\
- Pr{\"a}zisierung Filter Bots\\
- Bearbeitung Datensatz 2017, \textbf{Grafik Bots 2017}\\
- Grafik Mods\\
- Fragen zur Rekonstruktion\\

\subsection{Bots}
(Stichpunkte zu meiner Arbeit mit den Bots)
-erste Überlegung: Bots setzen genau alle 5 Minuten Pixel, um maximalen Nutzen zu erreichen \\
-Frage: sind Bots nur für einen einzigen Pixel oder für ein ganzes Gebiet verantwortlich? \\
-Recherche über die angewendeten Bot-Skripte ergibt tatsächlich: \\
--Bots wurde ein Bild als Referenz + Koordinaten überreicht. Innerhalb dieses Bereiches haben sie in festgelegten, gleichmäßigen Abständen Pixel des referenzierten Bildes gesetzt. \\
--Da sie stets in identischen Abständen Pixel setzten sind sie leicht zu erkennen (menschliche Nutzer könnten niemals mehrfach hintereinander auf die Sekunde genau sein) \\
--i.d.R waren diese Abstände größer als bloß 5 Minuten, da sie sonst als Spam eingestuft und gebannt werden würden \\
-Umsetzung: \\
--zeitlicher Abstand zwischen zwei aufeinanderfolgenden Pixeln als Differenz berechnet & mit dem Abstand zu den nächsten/vorherigen Pixeln verglichen \\
---Beispiel: \\
---Nutzer hat Pixel 1, 2, 3 & 4 in der Reihenfolge hintereinander gesetzt \\
---Wir bilden die zeitliche Differenz zwischen 1&2, 2&3, 3&4. Falls diese Differenzen identisch sind wird der Nutzer als Bot eingestuft. \\
---dies muss auf einen Nutzer nur ein einziges Mal im Ablauf des Experiments zutreffen, da die Zeiten in den Skripten angepasst werden können, bzw. der (Bot-)Account auch zwischendurch von einem menschlichen Nutzer bedient werden kann \\
-erfolgreich identifizierte Bots in 2022: ca. 17.500 (eventuell folgt präzisere Zahl), 110 Bots in 2017 \\
-(hier Bot-Statistiken 2017 vs 2022 (Farben, meistumkämpfte Pixel)) \\
-wie funktioniert das mit dem Nachzeichnen?: \\
--Liste mit den User_ids der Bots wird mit dem normalen Datensatz gejoined, damit wir die Daten zu den Pixeln nutzen können \\
--nacheinander wird jeder Pixel im Datensatz durchgegangen und auf einem Canvas gesetzt \\
---einzige Schwierigkeit: Farben werden im Datensatz als Hex-Werte gegeben und müssen vor Platzieren des Pixels in die richtigen RGB-Werte umgewandelt werden \\
-(Bot-Zeichnungen neben dem jeweils fertigen Place zeigen) \\

\subsection{Moderatoren}
Hier Input von Nicole!




\textbf{Abschluss}\\
- Statistiken 2022 und 2017 im Vergleich - Grafiken\\
- Vergleich Arbeit und Menge Bots\\
- Vergleich der Zensur von Mods