\section{Bots und Mods} \label{question2} % Change title accordingly

Statistische Auswertungen des Datensatzes:\\
\textit{Ist es m{\"o}glich Bots und Moderatoren anhand des Datensatzes zu identifizieren?}\\
\textit{K{\"o}nnen wir dazu entsprechende Statistiken entwickeln und auswerten?}\\
\textbf{17.05.}\\
Vermutungen:\\
- viele Nutzer setzen nur wenige Pixel, gro{\"ss}e H{\"a}ufigkeit bei kleiner Pixelzahl\\
- Bots setzen konstant viele Pixel, leicht erh{\"o}hte H{{"a}ufigkeit bei gro{\"ss}er Pixelzahl\\
- Testdatensatz umfasst eine Stunde, maximal 12 Pixel pro Nutzer m{\"o}glich\\
Vorgehen:\\
- Nach Nutzern aggregierte Pixeldaten nochmal nach Pixelanzahl aggregieren\\
- Grafik \\
\textbf{30.05.}\\
Statistiken:\\
- meiste umk{\"a}mpfter Pixel: \textbf{Grafik} ev. in 3-Dimensionen\\
- meiste verwendete Farben: \textbf{Grafik}\\
\textbf{21.06.}\\
- Filterkriterien Bots: ca. 14.000 Bots im Testdatensatz\\
- Filter Moderatoren,  Welche Farben wurden verwendet?\\
\textbf{28.06.}\\
- Grafik Bots 2022\\
\textbf{05.07.}\\
- Pr{\"a}zisierung Filter Bots\\
- Bearbeitung Datensatz 2017, \textbf{Grafik Bots 2017}\\
- Grafik Mods\\
- Fragen zur Rekonstruktion\\
\textbf{Abschluss}\\
- Statistiken 2022 und 2017 im Vergleich - Grafiken\\
- Vergleich Arbeit und Menge Bots\\
- Vergleich der Zensur von Mods