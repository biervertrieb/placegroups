\section{Identifikation der Nutzergruppen} \label{question1}  % Change title accordingly

Fragestellung 1: K{\"o}nnen anhand von {\"a}hnlichen Verhalten der Nutzer Nutzergruppen identifiziert werden?\\

\textbf{26.04.}\\
Betrachtung verschiedener $User_IDs$, die mehr als x-Mal zur selben Zeit und recht nah beieinander (Koordinaten) Pixel platziert/ ver{\"a}ndert haben\\
Zusätzlich anhand der gew{\"a}hlten Farben der User bzw. Nutzergruppen betrachten, ob zwei oder mehr Gruppen miteinander konkurrierten\\
\textbf{03.05.}\\
Definition Nutzergruppe:\\
- Nutzer ist zu verschiedenen Zeitintervallen aktiv\\
- Nutzer kann mehrere Interessen vertreten\\
- In jedem aktiven Zeitintervall vertritt ein Nutzer pro Raumzeitgebiet aber nur genau ein Interesse\\
- {\"U}berschneiden sich Interessen mehrerer Nutzer r{\"a}umlich und zeitlich, so stehen diese im Konflikt zueinander\\
- Beispiel Deutschland - Frankreich Flaggen\\
\textbf{10.05.}\\
- Aggregation Nutzerdaten\\
- Raumzeitmetrik (SVM)\\
- 3-dimensionalit{\"a}t der Daten, x-, y-Koordinate, Zeit\\
- Zusammenfassen von Raumzeiten (Bounding Boxes)\\
- Infrastrktur?\\
\textbf{17.05.}\\
- Bilder Bounding Boxes und zusammenh{\"a}ngende Pixel (Zusammenhangskomponenten) mit Beschreibung\\

\textbf{14.06.}
- Methodik weiteres Vorgehen\\

\textbf{Abschluss}\\
Was ist unser Abschluss, Ergebnis?


