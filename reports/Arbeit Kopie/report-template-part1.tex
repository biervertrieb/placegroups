\section{Introduction} \label{introduction}

Jedes Jahr zum 1. April f{\"u}hrt Reddit ein soziales Experiment durch. Im Jahr 2017 wurde dabei zum ersten Mal r/place vollkommen unangekündigt duchgeführt. Fünf Jahre später wurde das gleiche Experiment wiederholt, jedoch mit Vorankündigung. Bei r/place k{\"o}nnen Nutzer alle 5 Minuten einen Pixel auf einem Canvas setzen. In den Jahren 2017 und 2022 war der Canvas 1000 x 1000 Pixel bzw. 2000 x 2000 Pixel gro{\ss}. Das Experiment dauerte 4 Tage. \\
Nach erster Verwirrung der Nutzer zu Beginn des Experiments von 2017, bildeten sich einzelne Communities, welche gemeinsam Projekte verfolgten. Es entstanden erste Artworks und kreative Explosionen bis hin zu Kämpfen um bestimmte Regionen auf dem Canvas.\\
Vor allem bei der Wiederholung 2022 kamen vermehrt Scripts für Bots zum Einsatz, was vor allem der Vorankündigung geschuldet war, sodass Nutzer die Möglichkeit bekamen sich gezielt vorzubereiten. Anders als 2017, wurde die Wiederholung 2022 von Moderatoren begleitet, welche unerwünschte Artworks zensierten. \\
Da sich viele Communities vor allem 2022 zusammenschlossen, wurde die Frage, "ob es möglich ist, Nutzer anhand von ähnlicher Verhaltensweisen automatisch Nutzergruppen zuzuordnen?" zum Ziel dieses Projekts.\\
Weiterhin wurde auch betrachtet, ob Bots und Moderatoren identifizierbar sind und ob entsprechende Statistiken zu diesen entwickelt und ausgewertet werden können.
%Bilder vom Endcanvas 2017 und 2022 hier einfügen untereinander / nebeneinander?
\begin{figure}[t]
    \centering
    \includegraphics[width=\columnwidth]{rubber-duck}
    \caption{Some large rubber duck.}
    \label{rubber-duck}
\end{figure}

