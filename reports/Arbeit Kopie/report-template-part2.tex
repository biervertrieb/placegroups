\section{Data} \label{data}  % Change title accordingly

Der offizielle Datensatz von r/place aus der Wiederholung 2022 beinhaltet 72 Millionen Pixel, von 6 Millionen Nutzern und besitzt eine Größe von ca. 11GB. 
Dieser ist in einer CSV-Datei strukturiert, welche den Zeitpunkt des gesetzten Pixels (timestamp), eine gehashte Nutzeridentifikation ($user_id$), die verwendete Farbe ($pixel_color$) und die x- und y-Koordinaten auf dem Canvas (coordinates) beinhaltet. Eine Besonderheit stellen die Moderatoren im Datensatz dar. Diese besitzen in der Spalte coordinates nicht nur eine x- und y-Koordinate, wie alle anderen Nutzer, sondern zwei x- und y-Koordinaten. Das erste Koordinatenpaar bildet dabei die obere linke Ecke des Zensur-Quadrates eines Moderators und das zweite Koordinatenpaar die rechte untere Ecke. \\
Zu finden sind die Daten online unter: \url{https://www.reddit.com/r/place/comments/txvk2d/rplace_datasets_april_fools_2022/} und \url{https://www.reddit.com/r/redditdata/comments/6640ru/place_datasets_april_fools_2017/} für 2017.\\
Der Datensatz der ersten Durchführung 2017 ist mit einer Größe von ca. 1GB um einiges kleiner als der von 2022. Dieser hat eine ähnliche Gliederung wie der von 2022, jedoch wurden hier die x- und y-Koordinaten direkt getrennt aufgelistet statt unter einer gemeinsamen Spalte, wie in 2022. Zudem ist hier die Restriktion zu beachten, dass Nutzer nur zwischen 16 Farben wählen konnten und diese im Datensatz mit Integer-Values zwischen 0 bis 15 codiert wurden. Unter dem angegebenen Link für den Datensatz von 2017 findet sich eine entsprechende Tabelle, die wiedergibt, welcher Integer für welchen Farb-Hashwert steht. \\
Nachdem die Datensätze runtergeladen waren, mussten diese zunächst minimal bereinigt bzw. angepasst werden. Im Datensatz für 2017 mussten die Integer-Values der Farben in den jeweiligen Hashcode umgewandelt werden und für jeweils beide Datensätze wurden die Werte der timestamp-Spalte in nutzbare Werte umgewandelt werden. Dafür wurde das UTC-Timestampformat zunächst in das Unix-Epoch-Seconds-Format umgewandelt und anschließend der kleinste Wert von allen timestamps abgezogen. Damit waren die timestamps so normalisiert, dass der erste Eintrag mit einem timestamp gleich null beginnt. \\
Eine weitere Bereinigung der Daten beinhaltet, dass Moderatoren und "normale" Nutzer gespalten wurden, sodass diese in separaten DataFrames betrachtet werden können. Im Datensatz von 2022 wurde damit die Spalte coordinates jeweils in die x- und y-Koordinaten für "normale" Nutzer aufgespalten und für Moderatoren in x1- , y1- , x2- und y2-Koordinaten. Eine solche Spaltung musste für den 2017-Datensatz entsprechend nicht durchgeführt werden.
-  Beispieldaten\\

%Description of the datasets you used. You might want to mention: how acquired, how post-processed / cleaned, some basic characteristics, some examples from the data, etc.


